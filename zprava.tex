\documentstyle[epsf,epic,eepic,eepicemu]{article} \oddsidemargin=-5mm
\evensidemargin=-5mm\marginparwidth=.08in \marginparsep=.01in
\marginparpush=5pt\topmargin=-15mm\headheight=12pt
\headsep=25pt\footheight=12pt \footskip=30pt\textheight=25cm
\textwidth=17cm\columnsep=2mm
\columnseprule=1pt\parindent=15pt\parskip=2pt

\begin{document}
\begin{center}
\bf Semestralni projekt PAR 2007/2008:\\[5mm]
    Paralelni algoritmus pro reseni problemu UPO\\[5mm] 
       Miloslav Cinibulk\\
       Tomas Horacek\\[2mm]
5. rocnik, obor pocitace, K336 FEL CVUT, Karlovo nam. 13, 121 35 Praha 2\\[2mm]
\today
\end{center}

\section{Definice problemu a popis sekvencniho algoritmu}

Popiste problem, ktery vas program resi. Jako vychozi pouzijte text
zadani, ktery rozsirte o presne vymezeni vsech odchylek, ktere jste
vuci zadani behem implementace provedli 
(napr.  upravy heuristicke funkce, organizace zasobniku,
apod.). Zminte i pripadne i takove prvky algoritmu, ktere v zadani
nebyly specifikovany, ale ktere se ukazaly jako dulezite.  Dale
popiste vstupy a vystupy algoritmu (format vstupnich a vystupnich
dat). Uvedte tabulku namerenych casu sekvencniho algoritmu pro ruzne
velka data.

\section{Popis paralelniho algoritmu a jeho implementace v MPI}

Popiste paralelni algoritmus, opet vyjdete ze zadani a presne vymezte
odchylky, zvlaste u algoritmu pro vyvazovani zateze, hledani darce, ci
ukonceni vypoctu.  Popiste a vysvetlete strukturu celkoveho
paralelniho algoritmu na urovni procesuu v MPI a strukturu kodu
jednotlivych procesu. Napr. jak je naimplemtovana smycka pro cinnost
procesu v aktivnim stavu i v stavu necinnosti. Jake jste zvolili
konstanty a parametry pro skalovani algoritmu. Struktura a semantika
prikazove radky pro spousteni programu.

\section{Namerene vysledky a vyhodnoceni}

\begin{enumerate}
\item Zvolte tri instance problemu s takovou velikosti vstupnich dat, pro ktere ma sekvencni 
algoritmus casovou slozitost kolem 5, 10 a 15 minut.
Pro mereni cas potrebny na cteni dat z disku a ulozeni na disk neuvazujte a zakomentujte
ladici tisky, logy, zpravy a vystupy.
\item Merte paralelni cas pri pouziti $i=2,\cdot,24$ procesoru na sitich Ethernet a InfiniBand.
%\item Pri mereni kazde instance problemu na dany pocet procesoru spoctete pro vas algoritmus dynamicke delby prace celkovy pocet odeslanych zadosti o praci, prumer na 1 procesor a jejich uspesnost.
%\item Mereni pro dany pocet procesoru a instanci problemu provedte 3x a pouzijte prumerne hodnoty.
\item Z namerenych dat sestavte grafy zrychleni $S(n,p)$. Zjistete, zda a za jakych podminek
doslo k superlinearnimu zrychleni a pokuste se je zduvodnit.
\item Vyhodnodte komunikacni slozitost dynamickeho vyvazovani zateze a posudte
vhodnost vami implementovaneho algoritmu pro hledani darce a deleni zasobniku pri reseni vaseho
problemu. Posudte efektivnost a skalovatelnost algoritmu. Popiste nedostatky
vasi implementace a navrhnete zlepseni.
\item Empiricky stanovte 
granularitu vasi implementace, tj., stupen paralelismu pro danou velikost reseneho
problemu. Stanovte kriteria pro stanoveni mezi, za kterymi jiz neni
ucinne rozkladat vypocet na mensi procesy, protoze by komunikacni
naklady prevazily urychleni paralelnim vypoctem.

\end{enumerate}

\section{Literatura}

\appendix

\section{Navod pro vkladani grafu a obrazku do Texu}

Nejjednodussi zpusob vytvoreni obrazku je pouzit sunovsky graficky editor xfig,
ze ktrereho lze exportovat latex formaty (v poradi prosty latex, 
latex s macry epic, eepic, eepicemu) a postscript formaty,
uvedene poradi odpovida rustu komplikovanosti obrazku
(postscript umi jakykoliv obrazek, prosta latex macra pouze jednoduche,
epic makra neco mezi, je treba vyzkouset). Nasleduji priklady
pro vsechny pripady. 

Obrazek v postscriptu, vycentrovany a na celou sirku stranky, 
s popisem a cislem. Vsimnete si, jak ridit velikost obrazku.
% \begin{figure}[ht]
% \epsfysize=3cm
% \centerline{\epsfbox{VasObrazek.ps}}
% \caption{Popis vaseho obrazku}
% \label{labelvasehoobrazku}
% \end{figure}

Obrazek pouze vlozeny mezi radky textu, bez popisu a cislovani.\\
% \epsfxsize=1cm
% \rule{0pt}{0pt}\hfill\epsfbox{VasObrazek.ps}\hfill\rule{0pt}{0pt}

Texovske obrazky maji pripony *.latex, *.epic, *.eepic, a *.eepicemu, respective. 
% \begin{figure}[ht]
% \begin{center}
% \input VasObrazek.latex
% \end{center}
% \caption{Popis vaseho obrazku}
% \label{l1}
% \end{figure}
Vypustenim zavorek {\tt figure} dostanete opet pouze ramecek 
v textu bez cisla a popisu. 

Takhle jednoduse muzete poskladat obrazky vedle sebe.
% \begin{center}
% \setlength{\unitlength}{0.1mm}\input VasObrazek.epic
% \hglue 5mm 
% \setlength{\unitlength}{0.15mm}\input VasObrazek.eepic
% \hglue 5mm 
% \setlength{\unitlength}{0.2mm}\input VasObrazek.eepicemu
% \end{center}
Ridit velikost texovskych obrazku lze prikazem
\begin{verbatim}
\setlength{\unitlength}{0.1mm}
\end{verbatim}
ktere meni meritko rastru obrazku, Tyto prikazy je ale soucasne 
nutne vyhodit ze souboru, ktery xfig vygeneroval.

Pro vytvareni grafu lze pouzit program gnuplot, ktery umi generovat postscriptovy soubor, ktery vlozite
do Texu vyse uvedenym zpusobem.

\end{document}








